\documentclass{article}
\usepackage[14pt]{extsizes}
\usepackage[utf8x]{inputenc}
\usepackage[russian]{babel}
\usepackage{enumitem}
\usepackage{paralist}
\usepackage{setspace}
\usepackage[left=3cm,right=2cm,top=2cm,bottom=2cm,bindingoffset=0cm]{geometry}
\setstretch{1,5}

\begin{document}
\textbf{\large Система верстки ТеХ (LaTeX).}

ТеХ — это система компьютерной вёрстки, предназначенная для набора научно-техническихтекстов высокого полиграфического качества.
Пользователь задает текст и его структуру с помощью определенных команд на языке низкоуровневой разметки, а ТеХ сам форматирует документ.

LaTeX — наиболее популярный набор макрорасширений ТеХа.
Главная идея LaTeX состоит в том, что авторы должны думать о содержании, не беспокоясь о конечном визуальном облике. Готовя свой документ, автор указывает логическую структуру текста (разбивая его на главы, разделы, таблицы, изображения), а LaTeX решает вопросы его отображения. 
Это и является основным отличем TeX от MS Word.
В первом случае используется парадигма WYSIWYM (What You See Is What You Mean), а во втором  — WYSIWYG (What You See Is What You Get).

~\

\textbf{Преимущества LaТеХ:}
\begin{compactitem}
\item Высокое качество подготовки научных текстов, работы с формулами;
\item Автоматическая рубрикация документа, нумерация формул, рисунков и списка литературы;
\item Пользователь сосредоточится на структуре, а не оформлении;
\item Внесение изменений в визуальное представление документа не требует изменения самого документа.
\end{compactitem}

~\

\textbf{Недостатки LaTeX:}
\begin{compactitem}
\item Для работы с данным средством требуется знать язык разметки;
\item Затуднена работа с рисунками;
\item В MS Word более простой и понятный интерфейс: что мы видим, то и получаем.
\end{compactitem}
\end{document}